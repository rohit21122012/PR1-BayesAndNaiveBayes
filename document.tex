\documentclass[a4paper]{article}

\usepackage[english]{babel}
\usepackage[utf8]{inputenc}
\usepackage{amsmath,amsfonts}
\usepackage{graphicx}
\usepackage[colorinlistoftodos]{todonotes}
\usepackage[margin=1in]{geometry}
\usepackage{enumitem}

\setcounter{tocdepth}{5}
\setcounter{secnumdepth}{5}

\title{CS 669 Assignment 1}

\author{Rohit Patiyal \\ Devang Bacharwar}

\date{\today}

\begin{document}
\maketitle

\vspace{2.0cm}

\section {Objective}
	To build Bayes and Naive-Bayes classifiers for different types of data sets :
	\subsection{2-D artificial Data of 3 or 4 classes}
		\begin{enumerate}
		  \item {Linearly separable data set}
		  \item {Nonlinearly separable data sets (3 Data sets)}
		  \item {Overlapping data set}
		\end{enumerate}
	\subsection{Real World data set}

\vspace{2.0cm}

\section{Procedure}
	\begin{enumerate}
	  \item {Data for each class is partitioned into 75 \% for training and 25
	  \% for testing }
	  \item {Mean and Covariances are calculated for each class using the
	  training .}
	  \item {For points in a grid, likelihood is calculated for each class and is
	  labeled as of the class with the maximum likelihood probability.}
	  \par{For bayes classifier, the likelihood is assumed to be a multivariate
	  gaussian distribution }
	  \item {These labelled points are plotted with different colors to see the
	  different regions separated by the decision boundaries.}
	  \item {The testing data is also plotted over the regions, and observations a
	  re made.}
	\end{enumerate}
\vspace{2.0cm}

% \begin{figure}
% \begin{tabular}{cc}
%   \includegraphics[width=65mm]{ga.png} &   \includegraphics[width=65mm]{ga2.png}
%   \\
% (a) first & (b) second \\[6pt]
%  \includegraphics[width=65mm]{ga.png} &   \includegraphics[width=65mm]{ga2.png}
%  \\
% (c) third & (d) fourth \\[6pt]
% \multicolumn{2}{c}{\includegraphics[width=65mm]{ga.png} }\\
% \multicolumn{2}{c}{(e) fifth}
% \end{tabular}
% \caption{caption}
% \end{figure}

\section{Observations}
	\subsection{Bayes Classifier} 
		\subsubsection{Linearly separable data set}
		\centerline{\includegraphics[width=160mm,height=90mm]{plots/bayes/ls/same_cov.png}}
 		\centerline{\includegraphics[width=160mm,height=90mm]{plots/bayes/ls/avg_cov.png}}
 		\centerline{\includegraphics[width=160mm,height=90mm]{plots/bayes/ls/diff_cov.png}}
% 		\begin{figure}
% 		\begin{tabular}{c}
% 		  \includegraphics[width=100mm][height=50mm]{plots/lsbayes/LSBayesian_same_cov_final.png}
% 		  \\ (a) Same(all) Covariance \\
% 		 \includegraphics[width=100mm]{plots/lsbayes/LSBayesian_avg_cov_final.png}\\
% 		 (b) Same(avg) Covariance \\
% 		 \includegraphics[width=100mm]{plots/lsbayes/LSBayesian_diff_cov_final.png} \\
% 		 (c) Different Covariance \\%[6pt]
% 		 \includegraphics[width=65mm]{ga.png} &   \includegraphics[width=65mm]{ga2.png}
% 		 \\
% 		(c) third & (d) fourth \\[6pt]
% 		\multicolumn{2}{c}{\includegraphics[width=65mm]{ga.png} }\\
% 		\multicolumn{2}{c}{(e) fifth}
% 		\end{tabular}
% 		\caption{caption}
% 		\end{figure}
		\subsubsection{Non-Linearly separable data set }
			\paragraph{Data of Interlocking Classes}
			\centerline{\includegraphics[width=160mm,height=90mm]{plots/bayes/nls/interlock/same_cov.png}}
 			\centerline{\includegraphics[width=160mm,height=90mm]{plots/bayes/nls/interlock/avg_cov.png}}
 			\centerline{\includegraphics[width=160mm,height=90mm]{plots/bayes/nls/interlock/diff_cov.png}}
			\paragraph{A ring with a central mass}
			\centerline{\includegraphics[width=160mm,height=90mm]{plots/bayes/nls/ring/same_cov.png}}
 			\centerline{\includegraphics[width=160mm,height=90mm]{plots/bayes/nls/ring/avg_cov.png}}
 			\centerline{\includegraphics[width=160mm,height=90mm]{plots/bayes/nls/ring/diff_cov.png}}
			\paragraph{Spiral Dataset}
			\centerline{\includegraphics[width=160mm,height=90mm]{plots/bayes/nls/spiral/same_cov.png}}
 			\centerline{\includegraphics[width=160mm,height=90mm]{plots/bayes/nls/spiral/avg_cov.png}}
 			\centerline{\includegraphics[width=160mm,height=90mm]{plots/bayes/nls/spiral/diff_cov.png}}
		\subsubsection{Overlapping data set}
			\centerline{\includegraphics[width=160mm,height=90mm]{plots/bayes/over/same_cov.png}}
 			\centerline{\includegraphics[width=160mm,height=90mm]{plots/bayes/over/avg_cov.png}}
 			\centerline{\includegraphics[width=160mm,height=90mm]{plots/bayes/over/diff_cov.png}}
		\subsubsection{Real world data set}
			\centerline{\includegraphics[width=160mm,height=90mm]{plots/bayes/real/same_cov.png}}
 			\centerline{\includegraphics[width=160mm,height=90mm]{plots/bayes/real/avg_cov.png}}
 			\centerline{\includegraphics[width=160mm,height=90mm]{plots/bayes/real/diff_cov.png}}
	\subsection{Naive-Bayes classifier}
		\subsubsection{Linearly separable data set}
		\subsubsection{Non-Linearly separable data set }
			\paragraph{Data of Interlocking Classes}
			\paragraph{A ring with a central mass}
			\paragraph{Spiral Dataset}
		\subsubsection{Overlapping data set}
		\subsubsection{Real world data set}
\section{Conclusion}

\begin{verbatim}
> data=read.table("hw2_chol.txt")
> hist(data$V1,xlab='Cholesterol (mg/dL)',main='Histogram of Total Cholesterol')
> boxplot(data$V1,main='Total Cholesterol',ylab='Cholesterol (mg/dL)')
\end{verbatim}

\end{document}
              
            